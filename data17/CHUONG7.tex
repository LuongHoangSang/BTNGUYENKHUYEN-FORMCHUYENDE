\chapter{ĐỘNG LƯỢNG}
\begin{tomtat}
	\subsubsection{Động lượng}
	\paragraph{Khái niệm động lượng}
	\begin{dn}
		Động lượng là đại lượng đặc trưng cho khả năng truyền chuyển động của vật này lên vật khác thông qua tương tác giữa chúng.\\
		Động lượng của một vật được đo bằng tích của khối lượng và vận tốc của vật
		$$\vec{p}=m\vec{v}.$$
	\end{dn}
	Trong hệ SI, đơn vị của động lượng là \si{\kilogram\cdot\meter/\second}.
	\begin{note}
		\begin{itemize}
			\item Động lượng là đại lượng vector có hướng cùng hướng với hướng của vận tốc.
			\item Động lượng phụ thuộc vào hệ quy chiếu.
			\item Vector động lượng của nhiều vật bằng tổng các vector động lượng của các vật đó.
		\end{itemize}
	\end{note}
	\subsubsection{Định luật bảo toàn động lượng}
	\paragraph{Khái niệm hệ kín}
	\begin{dn}
		Một hệ được xem là \textbf{\textit{hệ kín}} khi hệ không có tương tác với các vật bên ngoài.\\
		Ngoài ra, một hệ được xem \textbf{\textit{gần đúng là hệ kín  }} nếu tương tác của các vật bên ngoài hệ (ngoại lực) lên hệ bị triệt tiêu hoặc ngoại lực không đáng kể so với lực tương tác giữa các thành phần của hệ (nội lực).
		
	\end{dn}
	\paragraph{Định luật bảo toàn động lượng}
	\begin{dl}
		Động lượng của một hệ kín luôn bảo toàn
		$$\vec{p}_1+\vec{p}_2+\dots+\vec{p}_n=\vec{p}^\prime_1+\vec{p}6\prime_2+\dots+\vec{p}_n.$$
	\end{dl}
	Trong đó:
	\begin{itemize}
		\item $\vec{p}_1+\vec{p}_2+\dots+\vec{p}_n$: tổng động lượng của hệ trước tương tác;
		\item $\vec{p}^\prime_1+\vec{p}^\prime_2+\dots+\vec{p}^\prime_n$: tổng động lượng của hệ sau tương tác.
	\end{itemize}
	\subsubsection{Mối liên hệ giữa lực tổng hợp tác dụng lên vật và tốc độ thay đổi động lượng}
	\begin{boxdn}
		Lực tác dụng lên vật bằng tốc độ thay đổi động lượng của vật
		$$\vec{F}=\dfrac{\Delta\vec{p}}{\Delta t}.$$
	\end{boxdn}
	hay
	\begin{boxdn}
	Độ biến thiên động lượng của một vật bằng xung lượng của lực (xung lực) tác dụng lên vật
		$$\Delta\vec{p}=\vec{F}\cdot\Delta t.$$
	\end{boxdn}
	Trong đó $\vec{F}\cdot\Delta t$ được gọi là xung lượng của lực (xung lực).
	\begin{note} Xung lực có cùng đơn vị với động lượng:
		\SI{1}{\kilogram\cdot\meter/\second}=\SI{1}{\newton\cdot\second}.
	\end{note}
	\subsubsection{Phân loại va chạm}
	\paragraph{Sự va chạm}
	\begin{dn}
		Va chạm là quá trình tương tác giữa hai vật có những đặc điểm sau:
		\begin{itemize}
			\item Thời gian tương tác rất ngắn.
			\item Lực tương tác (nội lực) rất lớn so với ngoại lực.
		\end{itemize}
		$\Rightarrow$ Hệ hai vật trên là một hệ kín.
	\end{dn}
	\paragraph{Va chạm đàn hồi - Va chạm mềm}
	\begin{center}
		\begin{tabular}{|M{3cm}|L{7cm}|L{7cm}|}
			\hline\rowcolor{gray!25!white}
			&\thead{Va chạm đàn hồi}&\thead{Va chạm mềm}\\
			\hline
			\thead{Đặc điểm}& \begin{itemize}
				\item Trong quá trình va chạm, hai vật bị biến dạng đàn hồi.
				\item Sau va chạm, hai vật lấy lại hình dạng ban đầu và chuyển động \textbf{tách rời nhau}.
			\end{itemize}& Sau va cham, hai vật \textbf{dính vào nhau} và chuyển động với \textbf{cùng vận tốc}.\\
			\hline
			\thead{Các định luật\\ bảo toàn}& \begin{itemize}
				\item Bảo toàn động lượng.
				\item Bảo toàn cơ năng (trường hợp va chạm tuyệt đối đàn hồi)$\Rightarrow$ Động năng của hệ trước và sau va chạm \textbf{bằng nhau}.
			\end{itemize}&\begin{itemize}
			\item Bảo toàn động lượng.
			\end{itemize}\begin{itemize}[label=\color{\mauly}\bfseries \faExclamationTriangle]
			\item Cơ năng \textbf{không bảo toàn}. Một phần động năng ban đầu chuyển hóa thành nhiệt năng, năng lượng liên kết hai vật lại với nhau, \dots $\Rightarrow$ Động năng lúc sau luôn \textbf{nhỏ hơn} động năng ban đầu.
			\end{itemize}\\
			\hline
		\end{tabular}
	\end{center}
	\begin{note}
		Mối liên hệ giữa động lượng và động năng:
		$$W_{\text{đ}}=\dfrac{p^2}{2m}\Rightarrow p=\sqrt{2mW_{\text{đ}}}$$
	\end{note}
\end{tomtat}