\chapter{NĂNG LƯỢNG}
%c \subsection{Tóm tắt lí thuyết}
\begin{tomtat}
	\subsubsection{Năng lượng}
	\paragraph{Tính chất của năng lượng}
	\begin{itemize}
		\item Năng lượng là đại lượng vô hướng.
		\item Năng lượng có thể tồn tại ở nhiều dạng khác nhau.
		\item Năng lượng có thể chuyển hóa qua lại giữa các dạng khác nhau và giữa các hệ, các thành phần của hệ.
		\item Trong hệ SI, đơn vị của năng lượng là joule $\left(\si{\joule}\right)$.
		\begin{note}
			1 calo là năng lượng cần thiết để làm cho \SI{1}{\gram} nước tăng lên thêm \SI{1}{\celsius}
			$$\SI{1}{cal}=\SI{4.18}{\joule}.$$
		\end{note}
	\end{itemize}
	\paragraph{Định luật bảo toàn và chuyển hóa năng lượng}
	\begin{dl}
		Năng lượng không tự nhiên sinh ra cũng không tự nhiên mất đi mà chỉ truyền từ vật này sang vật khác hoặc chuyển hóa từ dạng này sang dạng khác. Như vậy, năng lượng luôn được bảo toàn.
	\end{dl}
	\subsubsection{Công}
	\paragraph{Định nghĩa công}
	\begin{dn}
		Nếu lực không đổi $\vec{F}$ tác dụng lên vật và điểm đặt của lực đó chuyển dời một đoạn $s$ theo hướng hợp với hướng của lực góc $\alpha$ thì công của lực $\vec{F}$ được tính theo công thức:
		$$A=Fs\cos\alpha.$$
	\end{dn}
	Trong đó:
	\begin{itemize}
		\item $A$: công của lực $\vec{F}$ (\si{\joule});
		\item $F$: độ lớn lực tác dụng (\si{\newton});
		\item $s$: quãng đường vật dịch chuyển (\si{\meter});
		\item $\alpha$: góc hợp bởi lực $\vec{F}$ và chiều dịch chuyển.
	\end{itemize}
	\paragraph{Các trường hợp đặc biệt}
	\begin{itemize}
		\item Khi $\SI{0}{\degree}\leq \alpha<\SI{90}{\degree}$ thì $A>0$: lực thực hiện công dương hay công phát động.
		\item Khi $\alpha=\SI{90}{\degree}$ thì $A=0$: lực $\vec{F}$ không sinh công.
		\item Khi $\SI{90}{\degree}< \alpha\leq\SI{180}{\degree}$ thì $A<0$: lực thực hiện công âm hay công cản chuyển động.
	\end{itemize}
	\subsubsection{Công suất}
	\paragraph{Khái niệm công suất}
	\begin{dn}
		Công suất là đại lượng đặc trưng cho tốc độ sinh công của lực, được xác định bằng công sinh ra trong một đơn vị thời gian
		$$\calP=\dfrac{A}{t}.$$
	\end{dn}
\begin{note}
	Ngoài đơn vị watt (\si{\watt}) và các bội số của watt, một đơn vị thông dụng khác của công suất là mã lực:
	$$\SI{1}{HP}=\SI{746}{\watt}.$$
\end{note}
\paragraph{Mối liên hệ giữa công suất với lực tác dụng lên vật và vận tốc của vật}
\begin{boxdn}
	$$\calP_{\mathrm{tb}}=F\cdot v_{\mathrm{tb}}\cdot\cos\alpha.$$
\end{boxdn}
\subsubsection{Hiệu suất}
\begin{dn}
	Hiệu suất của động cơ là tỉ số giữa công suất có ích và công suất toàn phần của động cơ, đặc trưng cho hiệu quả làm việc của động cơ
	$$H=\dfrac{\calP_{\mathrm{ci}}}{\calP_{\mathrm{tp}}}\cdot\SI{100}{\percent}=\left(1-\dfrac{\calP_{\mathrm{hp}}}{\calP_{\mathrm{tp}}}\right)\cdot\SI{100}{\percent}.$$
\end{dn}
\subsubsection{Động năng}
\paragraph{Khái niệm}
\begin{dn}
	Động năng là dạng năng lượng của một vật có được do nó đang chuyển động 
	$$W_{\text{đ}}=\dfrac{1}{2}mv^2.$$
\end{dn}
\begin{note}
	\begin{itemize}
		\item Chỉ phụ thuộc độ lớn vận tốc, không phụ thuộc hướng vận tốc.
		\item Là đại lượng vô hướng, không âm.
		\item Mang tính tương đối.
	\end{itemize}
\end{note}

\paragraph{Định lý động năng}
\begin{hq}
	Độ biến thiên động năng bằng tổng công của các ngoại lực tác dụng vào vật:
	$$\dfrac{1}{2}mv^2-\dfrac{1}{2}mv^2_0=\sum A .$$
\end{hq}
\subsubsection{Thế năng trọng trường}
\paragraph{Khái niệm}
\begin{dn}
	Thế năng trọng trường của một vật là dạng năng lượng tương tác giữa Trái Đất và vật, nó phụ thuộc vào vị trí của vật trong trọng trường. Thế năng trọng trường của một vật có khối lượng $m$ đặt tại độ cao $h$ so với mốc thế năng là
	$$W_t=mgh.$$
\end{dn}
\paragraph{Độ giảm thế năng}
Công trọng lực bằng hiệu thế năng của vật tại vị trí đầu và vị trí cuối, tức là bằng độ giảm thế năng
$$A_{12}=A_1-A_2.$$
\begin{note}
	\begin{itemize}
		\item Thế năng có thể âm, dương hoặc bằng 0.
		\item Độ biến thiên thế năng giữa hai vị trí không phụ thuộc vào việc chọn gốc thế năng.
	\end{itemize}
\end{note}
\subsubsection{Cơ năng}
\paragraph{Khái niệm cơ năng}
\begin{dn}
	Cơ năng của vật chuyển động dưới tác dụng của trọng lực bằng tổng động năng và thế năng trọng trường của vật
	$$W=W_{\text{đ}}+W_{\text{t}}=\dfrac{1}{2}mv^2+mgh.$$
\end{dn}
\paragraph{Định luật bảo toàn cơ năng}
\begin{dl}
	Khi một vật chuyển động trong trọng trường chỉ chịu tác dụng của trọng lực thì cơ năng của vật là một đại lượng bảo toàn
	$$W=\dfrac{1}{2}mv^2+mgh=\text{hằng số}.$$
	hay
	$$\dfrac{1}{2}mv^2_1+mgh_1=\dfrac{1}{2}mv^2_2+mgh_2.$$
\end{dl}
Trong quá trình chuyển động của một vật trong trọng trường:
\begin{itemize}
	\item Nếu động năng của vật giảm thì thế năng của vật tăng và ngược lại.
	\item Tại vị trí có động năng cực đại thì thế năng cực tiểu và ngược lại.
\end{itemize}
\paragraph{Biến thiên cơ năng}
\begin{boxdl}
	Trong trường hợp vật chuyển động có sự xuất hiện của lực ma sát thì cơ năng không được bảo toàn. Khi đó, cơ năng ban đầu sẽ bằng tổng của của cơ năng lúc sau và phần năng lượng mất đi (thường là biến đổi thành nhiệt năng là công của lực ma sát)
	$$W=W^\prime+Q+\left|A\right|.$$
\end{boxdl}
Trong đó: 
\begin{itemize}
	\item $W$: cơ năng ban đầu (\si{\joule});
	\item $W^\prime$: cơ năng lúc sau (\si{\joule});
	\item $Q$: nhiệt lượng sinh ra (\si{\joule});
	\item $\left|A\right|$: độ lớn công của lực cản.
\end{itemize}
\end{tomtat}