%\renewenvironment{vd}{}{}
%\renewenvironment{bt}{}{}
%\renewenvironment{ex}{}{}
\renewenvironment{ex}[1][]{
	\def\ktghichu{#1}
	\def\khonghichu{}
	\ifx\ktghichu\khonghichu
	\par\hspace*{3cm} #1\par\vspace*{7mm}
	\else
	\par\hspace*{3cm} {\fontfamily{qag}\selectfont\bfseries\color{\mauEX}(#1)}\par\vspace*{2.5mm}
	\fi
}{}
%%
\renewenvironment{bt}[1][]{
	\def\ktghichu{#1}
	\def\khonghichu{}
	\ifx\ktghichu\khonghichu
	\par\hspace*{3cm} #1\par\vspace*{7mm}
	\else
	\par\hspace*{3cm} {\fontfamily{qag}\selectfont\bfseries\color{\mauEX}(#1)}\par\vspace*{2.5mm}
	\fi
}{}
%%
\renewenvironment{vd}[1][]{
	\def\ktghichu{#1}
	\def\khonghichu{}
	\ifx\ktghichu\khonghichu
	\par\hspace*{3cm} #1\par\vspace*{7mm}
	\else
	\par\hspace*{3cm} {\fontfamily{qag}\selectfont\bfseries\color{\mauEX}(#1)}\par\vspace*{2.5mm}
	\fi
}{}
%======================
\def\endbox{\end{tcolorbox}}
%%%%%%%%%%%%%%%%%%%%%%%
%Đưa chấm ra ngoài box
\renewcommand{\dotlinefull}[1]{
	\renewcommand{\chooseNSA}{\def\kindSA{}}
%	\def\tickT{}	% Chỗ này dùng để định nghĩa dấu tick đúng hoặc sai
%	\def\tickF{}
\AtBeginEnvironment{#1}{
	\def\colorEX{}%
	\chooseNSA
	\renewcommand{\TrueTF}{\FalseTF}
	\renewcommand{\TrueEX}{\FalseEX}
	\renewcommand{\writekeyTFone}{}
	\renewcommand{\writekeyTF}{&&}
	\renewcommand{\loigiai}[1]{
		\endbox
		\def\endbox{}
		\setlength{\paroutindent}{\expandafter\parindent}
		\setbox0=\vbox{
			\hbox{
				\noindent\begin{minipage}{\linewidth}%
					\setlength{\parindent}{\paroutindent}##1\end{minipage}
			}
		}
		\linepar=\ht0
		\pgfmathparse{round((\linepar-\fboxsep)/(\lineheight))}
		\par\noindent\textbf{\loigiaiEX}
		\dotlineEX{\pgfmathresult}
	}
}
}
		%-------------- Đóng khung câu ex
		\AtBeginEnvironment{ex}{
			\beginboxEX
			\renewcommand{\loigiai}[1]{
				\renewcommand{\immini}[2]{
					\setbox\imbox=\vbox{\hbox{##2}}
					\widthimmini=\wd\imbox
					\IMleftright{##1}{##2}}
				\endbox
				\begin{onlysolution}
					#1
				\end{onlysolution}
				\def\endbox{}
			}
		}
		\AtEndEnvironment{ex}{\endbox}	
		%-------------- Đóng khung câu bt
		\AtBeginEnvironment{bt}{
			\beginboxBT
			\renewcommand{\loigiai}[1]{
				\renewcommand{\immini}[2]{
					\setbox\imbox=\vbox{\hbox{##2}}
					\widthimmini=\wd\imbox
					\IMleftright{##1}{##2}}
				\endbox
				\begin{onlysolution}
					#1
				\end{onlysolution}
				\def\endbox{}
			}
		}
		\AtEndEnvironment{bt}{\endbox}
		%-------------- Ẩn đáp án
		\renewcommand{\indapan}[2]{}
		%---------- Ẩn lí thuyết
		%--Ẩn các mục con, chỉ để BÀI
		%\foreach \moitruong in{\subsection,\subsubsection,\chapter}{\renewcommand{\moitruong}[1]{}}
		\renewcommand{\subsection}[1]{}
		\renewcommand{\subsubsection}[1]{}
		\renewcommand{\chapter}[1]{}
		\renewcommand{\part}[1]{}
		%--Ẩn các môi trường
		%\hideenviron{dn}
		\hideenviron{note}
		\hideenviron{vd}
		\hideenviron{nx}
		\hideenviron{tomtat}
		\hideenviron{boxkn}
		\hideenviron{dang}
		%--ẩn bìa và mục lục
		\renewcommand{\includepdf}[1]{}
		\def\muclucchinh{}
		%------------
		\def\iconVD{}
		%------------
		%%------------
		\BeforeBeginEnvironment{bt}{
			\ifnum\the\value{BT}=0
			\par\dnsub{\fontfamily{qag}\selectfont\bfseries\stepcounter{stt}\thestt}
			\hspace*{1mm}
			{\fontfamily{qag}\selectfont\bfseries\color{\mausubsec}\MakeUppercase{\truocBT}}
			\fi
		}
%		\BeforeBeginEnvironment{dang}{
%			\ifnum\the\value{dang}=0
%			\par\dnsub{\fontfamily{qag}\selectfont\bfseries\stepcounter{stt}\thestt}
%			\hspace*{1mm}
%			{\fontfamily{qag}\selectfont\bfseries\color{\mausubsec}\MakeUppercase{\truocVD}}
%			\fi
%		}
		\BeforeBeginEnvironment{ex}{
			\ifnum\the\value{EX}=0
			\par\dnsub{\fontfamily{qag}\selectfont\bfseries\stepcounter{stt}\thestt}
			\hspace*{1mm}
			{\fontfamily{qag}\selectfont\bfseries\color{\mausubsec}\MakeUppercase{\truocEX}}
			\fi
		}
